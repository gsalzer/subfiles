% \iffalse meta-comment
%
% subfiles - class and package for multi-file projects in LaTeX
% Copyright 2002, 2012 Federico Garcia (feg8@pitt.edu, fedegarcia@hotmail.com)
% Copyright 2018-2020 Gernot Salzer (salzer@logic.at)
%
% This work may be distributed and/or modified under the
% conditions of the LaTeX Project Public License, either version 1.3
% of this license or (at your option) any later version.
% The latest version of this license is in
%   http://www.latex-project.org/lppl.txt
% and version 1.3 or later is part of all distributions of LaTeX
% version 2005/12/01 or later.
%
% This work has the LPPL maintenance status `maintained'.
% 
% The Current Maintainer of this work is Gernot Salzer.
%
% This work consists of the files subfiles.dtx and subfiles.ins
% and the derived files subfiles.cls, subfiles.sty and subfiles.pdf
%
% -------------------------------------------
%
%<*driver>
% \fi
\ProvidesFile{subfiles.dtx}[2020/09/07 v2.0 Multi-file projects]
% \iffalse
\documentclass{ltxdoc}
\GetFileInfo{subfiles.dtx}
\title{A Document Class and a Package\\for Handling Multi-File Projects}
\date{2020/09/07 v2.0}
\author{Federico Garcia, Gernot Salzer}
\usepackage{hyperref}
\begin{document}
\maketitle
 \DocInput{\filename}
\end{document}
%</driver>
% \fi
% \begin{abstract}
%   The |subfiles| package allows authors to split a document into one main file and several subsidiary files (subfiles) akin to the |\input| command, with the added benefit of making the subfiles compilable on their own.
%   This is achieved by reusing the preamble of the main file also for the subfiles.
% \end{abstract}
% \tableofcontents
% \section{Introduction}
% The \LaTeX\ commands |\include| and |\input| allow the user to split the \TeX\ source of a document into several input files.
% This is useful when creating documents with many chapters, but also for handling large tables, figures and code samples, which require a considerable amount of trial-and-errors.
%
% In this process the rest of the document is of little use, and can even interfere.
% For example, error messages may indicate not only the wrong line number, but may point to the wrong file.
% Frequently, one ends up wanting to work only on the new file:
% 
% \begin{itemize}
% \item Create a new file, and copy-paste the preamble of the main file into it.
% \item Work on this file, typeset it \emph{alone} as many times as necessary.
% \item Finally, when the result is satisfactory, delete the preamble from the file (alongside with |\end{document}|!), and |\include| or |\input| it from the main file.
% \end{itemize}
% 
% It is desirable to reduce these three steps to the interesting, middle one.
% Each new, subordinate file (henceforth `subfile') should behave both as a self-sufficient \LaTeX\ document and as part of the whole project, depending on whether it is \LaTeX ed individually or |\included|/|\input| from the main document.
% This is what the class |subfiles.cls| and the package |subfiles.sty| are intended for. 
% 
% \section{Basic usage}
%
% \DescribeMacro{subfiles.sty}
% The main file, i.e., the file with the preamble to be shared with the subfiles, has to load the package |subfiles|:
% \begin{center}
%   \begin{tabular}{l}
%     |\documentclass[...]{...}|\\
%     |\usepackage{subfiles}|\\
%     |\begin{document}|
%   \end{tabular}
% \end{center}
% \DescribeMacro{\subfile}
% Subordinate files (subfiles) are loaded from the main file or from other subfiles with the command
% \begin{center}
%   |\subfile{|\meta{subfile\_name}|}|
% \end{center}
% \DescribeMacro{subfiles.cls}
% The subfiles have to start with the line
% \begin{center}
%   |\documentclass[|\meta{main\_file\_name}|]{subfiles}|
% \end{center}
% which loads the class |subfiles|.
% Its only `option', which is actually mandatory, gives the name of the main file.
% This name follows \TeX\ conventions: |.tex| is the default extension, the path has to be provided if the main file is in a different directory, and directories in the path have to be separated by |/| (not |\|).
% Thus, we have the following structure.
% \begin{center}\small
%   \begin{tabular}[t]{l}
%     \multicolumn{1}{c}{main file} \\
%     \hline
%     |\documentclass[...]{...}| \\
%     \meta{shared preamble} \\
%     |\usepackage{subfiles}| \\
%     |\begin{document}| \\
%     \dots \\
%     |\subfile{|\meta{subfile\_name}|}| \\
%     \dots \\
%     |\end{document}| \\
%     \hline 
%   \end{tabular}
%   \hfill  
%   \begin{tabular}[t]{l}
%     \multicolumn{1}{c}{subfile} \\
%     \hline
%     |\documentclass[|\meta{main\_file\_name}|]{subfiles}| \\
%     |\begin{document}| \\
%     \dots \\
%     |\end{document}| \\
%     \hline
%   \end{tabular}
% \end{center}
% Now there are two possibilities.
% \begin{itemize}
% \item If \LaTeX\ is run on the subfile, the line |\documentclass[..]{subfiles}| is replaced by the preamble of the main file (including its |\documentclass| command).
%   The rest of the subfile is processed normally.
% \item If \LaTeX\ is run on the main file, the subfile is loaded like with an |\input| command, except that the preamble of the subfile up to  |\begin{document}| as well as |\end{document}| and the lines following it are ignored.
% \end{itemize}
%
% \section{Advanced usage}
%
% \subsection{Including files instead of inputting them}
%
% \DescribeMacro{\subfileinclude}
% In plain \LaTeX, you can use either |\input| or |\include| to load a file.
% In most cases the first is appropriate, but sometimes there are reasons to prefer the latter.
% Internally, the |\subfile| command uses |\input|.
% For those cases where you need |\include|, the package provides the command  
% \begin{center}
%   |\subfileinclude{|\meta{subfile\_name}|}|
% \end{center}
%
% \subsection{Fixing pathes}
%
% \DescribeMacro{\subfix}
% Whenever an error message of \LaTeX\ or an external program indicates that a file cannot be found, the reason may be that the missing file has to be addressed by varying pathes, depending on which file is typeset.
% In such a case, it may help to apply the command |\subfix| to the file or path names.
% Examples:
% \begin{center}
%   \begin{tabular}{ll}
%       package & command when used with |subfiles| \\
%       \hline
%       |biblatex| & |\addbibresource{\subfix{|\meta{file}|}}| \\
%       |bibunits| & |\putbib[\subfix{|\meta{file1}|},\subfix{|\meta{file2}|},|\dots|]|\\
%                  & |\defaultbibliography{\subfix{|\meta{file1}|},|\dots|}|
%     \end{tabular}
%   \end{center}
% \DescribeMacro{\bibliography}
% \DescribeMacro{\graphicspath}
% Some commands already apply the fix on the fly.
% At the moment these are the standard \LaTeX\ command |\bibliography| and |\graphicspath| from the |graphics|/|graphicx| package.
%
% \subsection{Conditional execution of commands}
%
% \DescribeMacro{\ifSubfilesClassLoaded}
% The command |\ifSubfilesClassLoaded| is useful to execute commands conditionally, depending on whether the main file is typeset or a subfile.
% \begin{center}
%   \begin{tabular}{l}
%     |\ifSubfilesClassLoaded{% then branch|\\
%       \quad \dots\ commands executed when the subfile is typeset \dots\\
%     |}{% else branch|\\
%       \quad \dots\ commands executed when the main file is typeset \dots\\
%     |}|
%   \end{tabular} 
% \end{center}
% As an example, this can be used to add the bibliography to the main document or to the subdocument, whichever is typeset:
% \begin{center}\small
%   \begin{tabular}[t]{l}
%     \multicolumn{1}{c}{main file} \\
%     \hline
%     |\documentclass[...]{...}| \\
%     |\usepackage{subfiles}|\\
%     |\bibliographystyle{alpha}|\\ 
%     |\begin{document}| \\
%       \dots \\
%     |\subfile{|\meta{subfile\_name}|}| \\
%       \dots \\
%     |\bibliography{bibfile}|\\
%     |\end{document}| \\
%     \hline 
%   \end{tabular}
%   \hfill  
%   \begin{tabular}[t]{l}
%     \multicolumn{1}{c}{subfile} \\
%     \hline
%     |\documentclass[|\meta{main\_file\_name}|]{subfiles}| \\
%     |\begin{document}| \\
%     \dots \\
%     |\ifSubfilesClassLoaded{%|\\
%     \quad|\bibliography{bibfile}%|\\
%     |}{}|\\
%     |\end{document}| \\
%     \hline
%   \end{tabular}
% \end{center}
%
% \subsection{Referencing between subfiles}
% When working with multiple subfiles under one main file, one often wants to refer in subfile |A.tex| to labels in subfile |B.tex|. To realize that, include
% in the preamble of the main file: 
% \begin{center}
%   |\usepackage{xr}|
% \end{center}
% Then, include in the preambles of |A.tex| and |B.tex|:
% \begin{center}
%  |\ifSubfilesClassLoaded{\externaldocument{|\meta{main\_file\_name}|}{}|
% \end{center}
%
% \subsection{Unusual locations for placing definitions and text}
%
% Starting with version 2.0, the |subfiles| package treats sub-preambles and text after |\end{document}| as one would expect:
% The preamble of subfiles is skipped when loaded with |\subfile|, and everything after |\end{document}| is ignored.
% In most cases this is what you want. 
%
% \DescribeMacro{v1}
% For reasons of compatibility, the option |v1| restores the behaviour of older versons:
% \begin{center}
%   |\usepackage[v1]{subfiles}|
% \end{center}
% This will have three effects.
%
% \emph{Code after the end of the main document} is added to the preamble of the subfiles, but is ignored when typesetting the main file.
% Here one can add commands that are to be processed as part of the preamble when the subfiles are typeset on their one.
% But this also means that any syntax error after |\end{document}| will ruin the \LaTeX ing of the subfile(s).
%
% \emph{Code in the preamble of a subfile} is processed as part of the text when typesetting the main file, but as part of the preamble when typesetting the subfile.
% This means that with the option |v1|, the preamble of a subfile can only contain stuff that is acceptable for both, the preamble and the text area.
% One should also keep in mind that each subfile is input within a group, so definitions made here may not work outside.
%
% \emph{Code after \texttt{\textbackslash end\{document\}} in a subfile} is treated like the code preceding it when the subfile is loaded from the main file, but is ignored when typesetting the subfile.
% The code after |\end{document}| behaves as if following the |\subfile| command in the main file, except that it is still part of the group enclosing the subfile.
% As a consequence, empty lines at the end of the subfile lead to a new paragraph in the main document, even if the |\subfile| command is immediately followed by text.
%
% \section{Use cases}
%
% \subsection{Hierarchy of directories}
%
% Sometimes it is desirable to put a subfile together with its images and supplementary files into its own directory.
% The difficulty now is that these additional files have to be addressed by different pathes depending on whether the main file or the subfile is typeset.
% As of version 1.3, the |subfiles| package handles this problem by using the |import| package.
%
% As an example, consider the following hierarchy of files:
% \begin{center}\ttfamily
%   \begin{tabular}{l}
%     main.tex\\
%     mypreamble.tex\\
%     dir1/subfile1.tex\\
%     dir1/image1.jpg\\
%     dir1/text1.tex\\
%     dir1/dir2/subfile2.tex\\
%     dir1/dir2/image2.jpg\\
%     dir1/dir2/text2.tex
%   \end{tabular}
% \end{center}
% where |main|, |subfile1|, and |subfile2| have the following contents:
% \begin{center}
% \begin{tabular}[t]{l}
%   \multicolumn{1}{c}{|main.tex|} \\
%   \hline
%   |\documentclass{article}| \\
%   |\input{mypreamble}| \\  
%   |\usepackage{graphicx}| \\
%   |\usepackage{subfiles}| \\
%   |\begin{document}| \\
%   |\subfile{dir1/subfile1}|\\
%   |\end{document}|\\
%   \hline
% \end{tabular}%
% \hfill
% \begin{tabular}[t]{l}
%   \multicolumn{1}{c}{|subfile1.tex|} \\
%   \hline
%   |\documentclass[../main]{subfiles}| \\
%   |\begin{document}| \\
%   |\input{text1}|\\ 
%   |\includegraphics{image1.jpg}|\\ 
%   |\subfile{dir2/subfile2}| \\  
%   |\end{document}| \\
%   \hline
% \end{tabular}\\[2ex]
% \begin{tabular}[t]{l}
%   \multicolumn{1}{c}{|subfile2.tex|} \\
%   \hline
%   |\documentclass[../../main]{subfiles}| \\
%   |\begin{document}| \\
%   |This is the example described in the manual, subfiles.pdf.
|\\ 
%   |\includegraphics{image2.jpg}|\\ 
%   |\end{document}| \\
%   \hline
% \end{tabular}
% \end{center}
% Then each of the three files can be typeset individually in its respective directory, where \LaTeX\ is able to locate all included text files and images.
%
%
% \subsection{Avoiding extra spaces}
%
% Sometimes you may want to load the contents of a subfile without white space separating it from the contents of the main file.
% In this respect, |\subfile| behaves similar to |\input|.
% Any space or newline before and after the |\subfile| command will appear in the typeset document, as will any white space between the last character of the subfile and |\end{document}|.
% Therefore, to load the contents of a subfile without intervening spaces, you have either to add comment signs:
% \begin{center}
%   \begin{tabular}[t]{l}
%       \multicolumn{1}{c}{|main.tex|}\\
%       \hline
%       \dots\\
%       |text before%|\\
%       |\subfile{sub.tex}%|\\
%       |text after|\\
%       \hline
%   \end{tabular}
%   \qquad
%   \begin{tabular}[t]{l}
%       \multicolumn{1}{c}{|sub.tex|}\\
%       \hline
%       |\documentclass[main.tex]{subfiles}|\\
%       |\begin{document}|\\
%       |contents of subfile%|\\
%       |\end{document}|\\
%       \hline
%   \end{tabular}
% \end{center}
% or to put everything on the same line:
% \begin{center}
%   |text before\subfile{sub.tex}text after|\\
%   |contents of subfile\end{document}|
% \end{center}
%
% \section{Troubleshooting}
%
% Here are some hints that solve most problems.
%
% \begin{enumerate}
% \item
%   Make sure to use the most recent version of the |subfiles| package, available from CTAN\footnote{\url{https://ctan.org/pkg/subfiles}} and Github\footnote{\url{https://github.com/gsalzer/subfiles}}.
% \item
%   Make sure that |\usepackage{subfiles}| appears near the end of the main preamble.
% \item
%   Make sure that the strings |\begin{document}| and |\end{document}| appear on lines of their own and that there are no additional characters preceding or trailing them.
% \item
%   If some external program that cooperates with \TeX, like |bibtex| or |biber|, complains about not being able to find a file, locate the name of the file in the \LaTeX\ source and replace \meta{filename} by |\subfix{|\meta{filename}|}|.
% \item
%   If nothing of the above helps, ask the nice people on tex.stackexchange\footnote{\url{https://tex.stackexchange.com/}}.
% \end{enumerate}
% 
% \section{Dependencies}
%
% The |subfiles| package uses the |import| package by Donald Arsenau to load subfiles from different directories.
% |import.sty| is part of the standard \TeX\ distribution.
%
% \section{Version history}
%
% \begin{description}
% \item[v1.1:]
%   Initial version by Federico Garcia. Subsequent versions by Gernot Salzer.
% \item[v1.2:] \mbox{}
%   \begin{itemize}
%   \item Incompatibility with classes and packages removed that modify the |\document| command, like the class |revtex4|.
%   \end{itemize}
% \item[v1.3:] \mbox{}
%   \begin{itemize}
%   \item Use of |import| package to handle directory hierarchies.
%   \item |\ignorespaces| added to avoid spurious spaces.
%   \item Incompatibility with commands removed that expect |\document| to be equal to |\@onlypreamble| after the preamble. Thanks to Eric Domenjoud for analysing the problem.
%   \end{itemize}
% \item[v1.4:] \mbox{}
%   \begin{itemize}
%   \item Incompatibility with |memoir| class and |comment| package removed.
%   \item Bug `|\unskip| cannot be used in vertical mode` fixed.
%   \end{itemize}
% \item[v1.5:] \mbox{}
%   \begin{itemize}
%   \item Command |\subfileinclude| added.
%   \item Basic support for |bibtex| related bibliographies in subfiles added.
%      Seems to suffice also for sub-bibliographies with the package |chapterbib|.
%   \item Support for sub-bibliographies with package |bibunits| added.
%   \end{itemize}
% \item[v1.6:] \mbox{}
%   \begin{itemize}
%   \item Support for sub-bibliographies with package |bibunits| dropped, in favor of |\subfix|.
%   \item Command |\subfix| added.
%   \item Incompatibility with |standalone| class removed.
%   \item The options of the main class are now also processed when typesetting a subfile; before they were ignored. Thanks to J\'an Kl'uka for analysing the problem.
%   \end{itemize}
% \item[v2.0:] \mbox{}
%   \begin{itemize}
%   \item Incompatibility with \LaTeX\ Oct.~2020 removed.
%     Thanks to Ulrike \mbox{Fischer} from the \LaTeX~3 team for the warning in time.
%   \item By default, text after |\end{document}| as well as the preamble of subfiles, when loaded with |\subfile|, are ignored now.
%     The old behaviour is available via the new package option |v1|.
%   \item Command |\ifSubfilesClassLoaded| added and documentation regarding the use of the |\bibliography| command corrected.
%     Thanks to Github user |alan-isaac| for reporting the issue.
%   \item Subfiles now can have the same name as the main file.
%     Thanks to Github user |June-6th| for reporting the issue.
%   \item Problem with the search path for images resolved.
%     Thanks to Github user |maxnick| for reporting the issue.
%   \end{itemize}
% \end{description}
%
%\section{The Implementation}
%\subsection{The class}
%    \begin{macrocode}
%<*class>
\NeedsTeXFormat{LaTeX2e}
\ProvidesClass{subfiles}[2020/09/07 v2.0 Multi-file projects (class)]
\DeclareOption*{%
  \typeout{Preamble taken from file `\CurrentOption'}%
  \let\preamble@file\CurrentOption
}
\ProcessOptions
%    \end{macrocode}
%
% After processing the option of the |subfiles| class, we reset |\@classoptionslist| such that the options in the main file will be processed.
%
%    \begin{macrocode}
\let\@classoptionslist\relax
%    \end{macrocode}
%
% To handle subfiles in separate directories, we use the |import| package.
% We load it now, since it resets the macro |\import@path|.
%    \begin{macrocode}
\RequirePackage{import}
%    \end{macrocode}
% 
% We redefine |\documentclass| to load the class of the main document.
% 
%    \begin{macrocode}
\let\subfiles@documentclass\documentclass
\def\documentclass{%
  \let\documentclass\subfiles@documentclass
  \LoadClass
}
%    \end{macrocode}
%
% In earlier versions, we used |\subimport| to load the preamble of the main file, which has the unwanted effect of undoing changes to the graphics path.
% Therefore we use |\input| and initialize |\import@path| and |\input@path| to the path of the main file.
% We use the internal \LaTeX\ macro |\filename@parse| to obtain this path.
%    \begin{macrocode}
\filename@parse{\preamble@file}
\edef\import@path{\filename@area}
\edef\input@path{{\filename@area}}
\input{\preamble@file}
%    \end{macrocode}
%
% After loading the preamble of the main file, we reset |\import@path{}|.
% Since the preamble may have changed the catcode of the |@| sign, we make it (again) a letter. Better safe than sorry.
%
%    \begin{macrocode}
{\makeatletter
  \gdef\import@path{}
}
%</class>      
%    \end{macrocode}
%
%
% \subsection{The package}
%    \begin{macrocode}
%<*package>
\NeedsTeXFormat{LaTeX2e}
\ProvidesPackage{subfiles}[2020/09/07 v2.0 Multi-file projects (package)]
%    \end{macrocode}
%
% The package has one option, |v1|, which affects the way how the text after |\end{document}| and in the preamble of subfiles is handled.
% With this option, |subfiles| behaves like in version 1.x.
% To implement the retro behaviour, we need a few definitions.
%
% First, we define three macros containing the strings |\begin{document}|, |\end{document}| and |document|.
% We need them later to detect these strings in the input.
%    \begin{macrocode}
\def\subfiles@DOCUMENT{document}
{\escapechar=-1\relax
 \xdef\subfiles@BEGINDOCUMENT{\string\\begin\string\{document\string\}}%
 \xdef\subfiles@ENDDOCUMENT{\string\\end\string\{document\string\}}%
}
%    \end{macrocode}
%
% The macro |\subfiles@skipDocument| skips everything until it encounters the string |\end{document}|.
% In earlier versions of the package, this was accomplished by redefining the |document| environment to the |comment| environment of the |verbatim| package.
% This clashes with the \LaTeX\ format published in October 2020, so we copy and adapt the code from the |verbatim| package (see the definition of |\subfiles@skiplines| below).
%
%    \begin{macrocode}
\def\subfiles@skipDocument{%  
  \let\subfiles@skiplinescont\ignorespaces
  \let\subfiles@skiplinesend\subfiles@ENDDOCUMENT
  \subfiles@skiplines
}
%    \end{macrocode}
%
% To skip |\documentclass| the old way, we redefine it to do nothing except restoring its original definition.
%
%    \begin{macrocode}
\def\subfiles@skipDocumentclass{%
  \renewcommand\documentclass[2][]{%
    \let\documentclass\subfiles@documentclass
    \ignorespaces
  }%
}
%    \end{macrocode}
%
% To ignore the preamble of a subfile, we redefine |\documentclass| to skip everything until the string |\begin{document}| is encountered.
%    \begin{macrocode}
\def\subfiles@skipPreamble{%
  \def\documentclass{%
    \let\documentclass\subfiles@documentclass
    \def\subfiles@skiplinescont{\begin{document}}%
    \let\subfiles@skiplinesend\subfiles@BEGINDOCUMENT
    \subfiles@skiplines
  }%
}
%    \end{macrocode}
%
% Now we set the default behaviour of the subfiles package:
% When loading the main preamble in a subfile, everything after the preamble is ignored.
% Moreover, when reading a subfile, its preamble as well as everything after |\end{document}| is ignored.
%    \begin{macrocode}
\let\subfiles@handleMain\endinput
\let\subfiles@handleSubpreamble\subfiles@skipPreamble
\let\subfiles@handleTextAfterSubdocument\endinput
%    \end{macrocode}
%
% The option |v1| restores the behaviour of the old |subfiles| package: When loading the main preamble in a subfile, only the contents of the |document| environent is ignored, but not the stuff following it. Moreover, when reading a subfile, only the |\documentclass| command and the lines |\begin{document}| and |\end{document}| are ignored, but the subfile preamble as well as everything after |\end{document}| is retained.
%    \begin{macrocode}
\DeclareOption{v1}{%
  \let\subfiles@handleMain\subfiles@skipDocument
  \let\subfiles@handleSubpreamble\subfiles@skipDocumentclass
  \let\subfiles@handleTextAfterSubdocument\relax
}
\DeclareOption*{\PackageWarning{subfiles}{Option '\CurrentOption' ignored}}
\ProcessOptions\relax
%    \end{macrocode}
% 
% To skip everything until a specific string is read, we adapt code from the |verbatim| package.
% The skipping of lines is controlled by two macros that have to be set before calling |\subfiles@skiplines|.
% |\subfiles@skiplinesend| is the string that marks the end of the skipped area; it has to appear on a line of its own, as the only content of this line.
% |\subfiles@skiplinescont| contains the code to be executed after skipping has ended.
%  
%    \begin{macrocode}
\def\subfiles@skiplines{%
  \begingroup
    \let\do\@makeother\dospecials
    \@makeother\^^L%
    \endlinechar`\^^M\relax \catcode`\^^M=12\relax \subfiles@skipline}
{\catcode`\^^M=12 \endlinechar=-1 %
 \gdef\subfiles@skipline#1^^M{\def\subfiles@tmp{#1}%
      \ifx\subfiles@tmp\subfiles@skiplinesend
          \def\subfiles@tmp{\endgroup\subfiles@skiplinescont}%
      \else\let\subfiles@tmp\subfiles@skipline
      \fi \subfiles@tmp}
}
%    \end{macrocode}
%
% To handle subfiles in separate directories, we use the |import| package.
% If it has already been loaded, e.g.\ by the |subfiles| class, this line does nothing.
%    \begin{macrocode}
\RequirePackage{import}
%    \end{macrocode}
% 
% The |\subimport| command requires path and filename as separate arguments, so we have to split file locations into these two components.
% The internal \LaTeX\ command |\filename@parse| almost fits the bill, except that it additionally splits the filename into basename and extension.
% Unfortunately, concatenating basename and extension to recover the filename is not clean:
% Under Unix/Linux, the filenames |base| and |base.| denote different entities, but after |\filename@parse| both have the same basename and an empty extension.
% Therefore we redefine the command |\filename@simple| temporarily; it is responsible for this unwanted split.
%  
%    \begin{macrocode}
\def\subfiles@split#1{%
  \let\subfiles@filename@simple\filename@simple
  \def\filename@simple##1.\\{\edef\filename@base{##1}}%
  \filename@parse{#1}%
  \let\filename@simple\subfiles@filename@simple
}%
%    \end{macrocode}
%
% E.g., after executing |\subfiles@split{../dir1/dir2/file.tex}| the macros |\filename@area| and |\filename@base| expand to |../dir1/dir2/| and |file.tex|, respectively.
%
% \DescribeMacro{\subfile}
% The command |\subfile| specifies the command |\subimport| for |\input|ing the subfile, and then calls |\subfiles@subfile|.
%
%    \begin{macrocode}
\newcommand\subfile{%
  \let\subfiles@loadfile\subimport
  \subfiles@subfile
}
%    \end{macrocode}
%
% \DescribeMacro{\subfileinclude}
% The command |\subfileinclude| specifies the command |\subincludefrom| for |\include|ing the subfile, and then calls |\subfiles@subfile|.
%
%    \begin{macrocode}
\newcommand\subfileinclude{%
  \let\subfiles@loadfile\subincludefrom
  \subfiles@subfile
}
%    \end{macrocode}
%
% The main functionality of the two commands is implemented in |\subfiles@subfile|.
% It redefines |\documentclass| and the |document| environment to do nothing but reverting these command to their original meaning and avoiding spurious spaces.
% Reverting |\documentclass| and |\document| to their original definition is important for being compatible with classes like |standalone| or packages like |bibentry|, which rely on this definition.
%
%    \begin{macrocode}
\newcommand\subfiles@subfile[1]{%
  \begingroup
  \let\subfiles@documentclass\documentclass
  \let\subfiles@document\document
  \let\subfiles@enddocument\enddocument
  \subfiles@handleSubpreamble
  \renewenvironment{document}{%
    \let\document\subfiles@document
    \ignorespaces
  }{%
    \let\enddocument\subfiles@enddocument
    \@ignoretrue
    \subfiles@handleTextAfterSubdocument
  }%
%    \end{macrocode}
%
% Now we split the file name into path and base name and load the file.
%
%    \begin{macrocode}
  \subfiles@split{#1}%
  \subfiles@loadfile{\filename@area}{\filename@base}%
  \endgroup
}
%    \end{macrocode}
%
% \DescribeMacro{\subfix}
% If some package provides a command that takes a filename as argument, then it has to be prefixed with the current |\import@path|.
% This is what the |\subfix| command tries to do.
% In order to succeed, the filename has to be expanded immediately, such that the current value of |\import@path| is used.
%
%    \begin{macrocode}
\def\subfix#1{\import@path#1}
%    \end{macrocode}
%
% For patching a list of file or path names, we define two auxiliary macros, one iterating over a comma-separated list of names and one processing a sequence of names enclosed in braces.
%
%    \begin{macrocode}
\def\subfiles@fixfilelist#1{%
  \def\subfiles@list{}%
  \def\subfiles@sep{}%
  \@for\subfiles@tmp:=#1\do{%
    \edef\subfiles@list{\subfiles@list\subfiles@sep\subfix{\subfiles@tmp}}%
    \def\subfiles@sep{,}%
  }%
}
\def\subfiles@fixpathlist#1{%
  \def\subfiles@list{}%
  \@tfor\subfiles@tmp:=#1\do{%
    \edef\subfiles@list{\subfiles@list{\subfix\subfiles@tmp}}%
  }%
}
%    \end{macrocode}
%
% \DescribeMacro{\bibliography}
% \DescribeMacro{\graphicspath}
% We patch |\bibliography| and |\graphicspath| (from the |graphics|/|graphicx| package) such that users don't have to worry about adding |\subfix|.
%
%    \begin{macrocode}
\let\subfiles@bibliography\bibliography
\renewcommand\bibliography[1]{%
  \subfiles@fixfilelist{#1}%
  \expandafter\subfiles@bibliography\expandafter{\subfiles@list}%
}
\@ifpackageloaded{graphics}{%
  \let\subfiles@graphicspath\graphicspath
  \renewcommand\graphicspath[1]{%
    \subfiles@fixpathlist{#1}%
    \edef\subfiles@list{{\subfix{}}\subfiles@list}%
    \expandafter\subfiles@graphicspath\expandafter{\subfiles@list}%
  }%
}{}
%    \end{macrocode}
%
% \DescribeMacro{\ifSubfilesClassLoaded}
% To add code or text conditionally, depending on whether the main document or a subfile is typeset, we provide the command |\ifSubfilesClassLoaded|.
%
%    \begin{macrocode}
\newcommand\ifSubfilesClassLoaded{%
  \expandafter\ifx\csname ver@subfiles.cls\endcsname\relax
    \expandafter\@secondoftwo
  \else
    \expandafter\@firstoftwo
  \fi
}
%    \end{macrocode}
%
% The |subfiles| package is loaded near the end of the main preamble.
% If it is loaded from a subfile, i.e., if |subfiles.cls| has been loaded, then we have to prepare for skipping the main document.
% We do this be redefining the |\begin| command.
% Normally, the first |\begin| after the |subfiles| package starts the main document.
% To allow for the case that some other environment occurs before (does it really happen?) we test whether we are dealing with |\begin{document}|.
% If not, we execute the original definition of |\begin|; otherwise we skip the main document as specified by |\subfiles@handleMain|.
%
%    \begin{macrocode}
\ifSubfilesClassLoaded{%
  \let\subfiles@begin\begin
  \def\begin#1{%
    \def\subfiles@tmp{#1}%
    \ifx\subfiles@tmp\subfiles@DOCUMENT
      \let\begin\subfiles@begin
      \let\subfiles@tmp\subfiles@handleMain
    \else 
      \def\subfiles@tmp{\subfiles@begin{#1}}%
    \fi
    \subfiles@tmp
  }%
}{}
%</package>
%    \end{macrocode}
